\documentclass[twocolumn,landscape,8pt]{article}
\usepackage{lipsum}% http://ctan.org/pkg/lipsum
\usepackage{graphicx,dblfloatfix}% http://ctan.org/pkg/{graphicx,dblfloatfix}
\usepackage{longtable}
\usepackage{tikz}
\usepackage{blindtext}
\usepackage{calendar} % Use the calendar.sty style
\usepackage[condensed,math]{anttor}
\usepackage[T1]{fontenc}
\usepackage{pdfpages}


\usepackage[landscape,margin=0.5in]{geometry}
\newcommand\framethispage[1][1cm]{%
    \tikz[overlay,remember picture,line width=1pt]
    \draw([xshift=(#1),yshift=(-#1)]current page.north west)rectangle
         ([xshift=(-#1),yshift=(#1)]current page.south east);%
}

\usepackage{catchfile}
\newcommand{\getenv}[2][]{%
  \CatchFileEdef{\temp}{"|kpsewhich --var-value #2"}{}%
  \if\relax\detokenize{#1}\relax\temp\else\let#1\temp\fi}

\begin{document}
\framethispage[0.2cm]% le cadre est à 2 cm des bords de la feuille

\getenv[\RES]{RES}\show\RES

\title{Finance}

%Example de la police que je voulais

\makeatletter
\newcommand\CTFont[1][small]{
\renewenvironment{table}
               {\@float{table}\csname#1\endcsname}
               {\end@float}
\renewenvironment{table*}
               {\@dblfloat{table}\csname#1\endcsname}
               {\end@dblfloat}
}
\makeatother

\maketitle

%\blindtext
%    \begin{tikzpicture}[remember picture,overlay]
%    \node[anchor=north west,yshift=-15pt,xshift=15pt]%
%        at (current page.north west)
%        {\includegraphics[height=15mm]{Logo}};
%    \end{tikzpicture}
%\blindtext

%{\footnotesize
%Blalalallalalal/home/frederickerdraon/Documents/researchwork/Latex/
%}

%\begin{figure*}[b]
%  \centering\includegraphics[width=\linewidth,height=0.3\textheight]{cal1}
%  \caption{B caption}
%\end{figure*}

%\begin{figure*}[t]
%  \centering\includegraphics[width=\linewidth,height=0.3\textheight]{burn}
%  \caption{B caption}
%\end{figure*}

%\lipsum[4-7]

\begin{figure}[t]
  \centering\includegraphics[width=\columnwidth,height=0.1\textheight]{linear}
  %\caption{A caption}
\end{figure}

%{\footnotesize
%Blalalallalalal
%}

%Mon texte de base ici
\begin{figure}[t]
  \centering\includegraphics[width=\columnwidth,height=0.2\textheight]{cash}
  %\caption{A caption}
\end{figure}

\begingroup
%\CTFont% tables in \small size
\CTFont[tiny]% tables in \tiny size
\begin{figure}[t]
{\footnotesize
\input{$RES/Latex/cashflows}
}
\end{figure}
\endgroup

\begin{figure}[t]
{\footnotesize
\input{/home/frederickerdraon/Documents/researchwork/Latex/contacts}
}
\end{figure}

\begin{figure}[t]
  \centering\includegraphics[width=\columnwidth,height=0.3\textheight]{pnl}
  %\caption{A caption}
\end{figure}

%\begin{figure}[t]
%  \centering\includegraphics[width=\columnwidth,height=0.3\textheight]{contacts}
  %\caption{A caption}
%\end{figure}

%\begin{figure}[t]
%  \centering\includegraphics[width=\columnwidth,height=0.3\textheight]{tasks}
  %\caption{A caption}
%\end{figure}

\begin{figure}[t]
{\footnotesize
\input{/home/frederickerdraon/Documents/researchwork/Latex/tasks}
}
\end{figure}

%\begin{figure}[t]
%{\footnotesize
%\input{/home/frederic/researchwork/MyFirstWindow/Latex/events}
%}
%\end{figure}

%\begin{figure}[t]
%{\footnotesize
%\input{/home/frederic/researchwork/MyFirstWindow/Latex/chargesCheese}
%}
%\end{figure}

%\begin{figure}[t]
%  \centering\includegraphics[width=\columnwidth,height=0.1\textheight]{Brownian}
%  \caption{A caption}
%\end{figure}
%Tentative 
\begin{figure}[t]
  \centering\includegraphics[width=\columnwidth,height=0.1\textheight]{burndown}
  %\caption{A caption}
%\input{BarPlot}
\end{figure}

%\begin{figure*}[b]
%  \centering\includegraphics[width=\linewidth,height=0.6\textheight]{cal}
%  %\caption{B caption}
%\end{figure*}

%\begin{figure}[t]
%{\footnotesize
%\input{/home/frederic/researchwork/MyFirstWindow/Latex/calendar_3}
%}
%\end{figure}


\begin{figure}[t]
  \centering\includegraphics[width=\columnwidth,height=0.1\textheight]{/home/frederic/researchwork/MyFirstWindow/Latex/tasks}
  \caption{A caption}
\end{figure}

\begin{figure}[t]
{\footnotesize
  \input{/home/frederickerdraon/Documents/researchwork/Latex/cashflows_other}
}
\end{figure}

% Ici on change pour ne mettre les éléments que sur une seule colonne, c'est à dire la page complète
% Ici on change pour ne mettre les éléments que sur une seule colonne, c'est à dire la page complète
\begingroup
\onecolumn
\begin{figure}[t]
  \centering\includegraphics[width=\columnwidth,height=1.0\textheight]{TimeTable}
  \caption{A caption}
\end{figure}
\endgroup

\begingroup
\onecolumn
\begin{figure}[t]
\includepdf[pages={1}]{/home/frederickerdraon/researchwork/Latex/Calendar17.pdf}
\end{figure}
\endgroup

%\onecolumn
%\begingroup
%\centering
%{\LARGE The Title \\[1.5em]
%\large First Author\Mark{1}, Second Author\Mark{2}, Third Author\Mark{1}, Fourth Author\Mark{2} and Fifth Author\Mark{3}}\\[1em]
%\begin{tabular}{*{2}{>{\centering}p{.25\textwidth}}}
%\Mark{1}Department1 & \Mark{2}Department2 \tabularnewline
%School1 & School2  \tabularnewline
%\url{email1} & \url{email2}
%\end{tabular}\par
%\twocolumn
%\endgroup

%\begin{figure}[t]
%  \centering\includegraphics[width=\columnwidth,height=0.1\textheight]{expenses}
  %\caption{A caption}
%\end{figure}

%\lipsum[8-12]
\end{document}
